\documentclass[modern]{aastex631}

% Packages
\usepackage{microtype}  % ALWAYS!
\usepackage{amsmath}
\usepackage{amsfonts}
\usepackage{amssymb}

% Style tweaks
\renewcommand{\twocolumngrid}{\onecolumngrid}
\setlength{\parindent}{1.1\baselineskip}
\sloppy\sloppypar\raggedbottom\frenchspacing


%%%%%%%%%%%%%%%%%%%%%%%%%%%%%%%%%%%%%%%%%%%%%%%%%%%%%%%%%%%%%%%%%%%%%%%%%%%%%%%%
\shorttitle{Timing argument blah blah}
\shortauthors{Chamberlain, Price-Whelan et al.}

%%%%%%%%%%%%%%%%%%%%%%%%%%%%%%%%%%%%%%%%%%%%%%%%%%%%%%%%%%%%%%%%%%%%%%%%%%%%%%%%
\graphicspath{{./}{figures/}}
% Missions
\newcommand{\project}[1]{\textsl{#1}}

% Packages / projects / programming
\newcommand{\package}[1]{\textsl{#1}}
\newcommand{\acronym}[1]{{\small{#1}}}
\newcommand{\github}{\package{GitHub}}
\newcommand{\python}{\package{Python}}
\newcommand{\astropy}{\package{Astropy}}

% Stats / probability
\newcommand{\given}{\,|\,}
\newcommand{\norm}{\mathcal{N}}
\newcommand{\pdf}{\textsl{pdf}}

% Maths
\newcommand{\dd}{\mathrm{d}}
\newcommand{\transpose}[1]{{#1}^{\mathsf{T}}}
\newcommand{\inverse}[1]{{#1}^{-1}}
\newcommand{\argmin}{\operatornamewithlimits{argmin}}
\newcommand{\mean}[1]{\left< #1 \right>}

% Non-scalar variables
\renewcommand{\vec}[1]{\ensuremath{\bs{#1}}}
\newcommand{\mat}[1]{\ensuremath{\mathbf{#1}}}

% Unit shortcuts
\newcommand{\Msun}{\ensuremath{\mathrm{M}_\odot}}
\newcommand{\Mjup}{\ensuremath{\mathrm{M}_{\mathrm{J}}}}
\newcommand{\kms}{\ensuremath{\mathrm{km}~\mathrm{s}^{-1}}}
\newcommand{\pc}{\ensuremath{\mathrm{pc}}}
\newcommand{\kpc}{\ensuremath{\mathrm{kpc}}}
\newcommand{\Mpc}{\ensuremath{\mathrm{Mpc}}}
\newcommand{\kmskpc}{\ensuremath{\mathrm{km}~\mathrm{s}^{-1}~\mathrm{kpc}^{-1}}}
\newcommand{\dayd}{\ensuremath{\mathrm{d}}}
\newcommand{\yr}{\ensuremath{\mathrm{yr}}}
\newcommand{\Kel}{\ensuremath{\mathrm{K}}}

% Misc.
\newcommand{\bs}[1]{\boldsymbol{#1}}

% Astronomy
\newcommand{\DM}{{\rm DM}}
\newcommand{\feh}{\ensuremath{{[{\rm Fe}/{\rm H}]}}}
\newcommand{\df}{\acronym{DF}}

% TO DO
\newcommand{\todo}[1]{{\color{red} TODO: #1}}
\newcommand{\apw]}[1]{{\color{green} APW says: #1}}

% Projects
\newcommand{\gaia}{\textsl{Gaia}}
\newcommand{\gaiadr}{\textsl{Gaia}~\acronym{EDR3}}


% Affiliations
\newcommand{\affuofa}{University of Arizona, 933 N. Cherry Ave,
    Tucson, AZ 85721, USA}
\newcommand{\affcca}{Center for Computational Astrophysics, Flatiron Institute,
    Simons Foundation, 162 Fifth Avenue, New York, NY 10010, USA}

\newcommand{\kc}[1]{\textcolor{mypink}{\textbf{#1}} }


%% This is the end of the preamble.  Indicate the beginning of the
%% manuscript itself with \begin{document}.

\begin{document}

\title{
    A timing argument mass for the Local Group accounting for the
    Milky Way--LMC reflex motion
}

\author[0000-0001-8765-8670]{Katie~Chamberlain}
\affiliation{\affuofa}

\author[0000-0003-0872-7098]{Adrian~M.~Price-Whelan}
\affiliation{\affcca}

\author{Others!}


\begin{abstract}



\end{abstract}

\section{Introduction}
\label{sec:intro}

\begin{itemize}
    \item importance of knowing mass of the local group
    \begin{itemize}
            \item 
            \item 
          \end{itemize}
    \item ways to measure local group mass
    \item \begin{itemize}
            \item other tracers? 
            \item introduction of timing argument 
            \item what previous TA measurements of the local group mass have been
          \end{itemize}
    \item what's wrong with previous use of TA?
        \begin{itemize}
            \item the LMC is massive and pulling on the MW disk
            \item previous use of timing argument did not account for travel velocity of the disk
            \item we include the travel velocity of the disk
        \end{itemize}
    \item cosmological impact
        \begin{itemize}
            \item bias and calibration of the timing argument 
            \item what a lower/higher group mass means (we're gonna be more consistent w other group mass measurements, I think?)
        \end{itemize}
    
\end{itemize}


\section{The timing argument as a Local Group mass estimator}
\label{sec:timing}


\section{Data analysis and methods}
\label{sec:methods}


\section{Results: Local group mass estimates}
\label{sec:results}


\section{Comparison to M33--M31 system}
\label{sec:results}


\section{Summary and Discussion}
\label{sec:discussion}


\software{
    Astropy \citep{astropy, astropy:2018},
    gala \citep{gala},
    IPython \citep{ipython},
    numpy \citep{numpy},
    % pymc3 \citep{Salvatier2016},
    scipy \citep{scipy}
}

\appendix
For baddies only.

\bibliography{refs}{}
\bibliographystyle{aasjournal}

\end{document}
